% Created 2021-11-19 Fri 13:20
% Intended LaTeX compiler: pdflatex
\documentclass[11pt]{article}
\usepackage[utf8]{inputenc}
\usepackage[T1]{fontenc}
\usepackage{graphicx}
\usepackage{grffile}
\usepackage{longtable}
\usepackage{wrapfig}
\usepackage{rotating}
\usepackage[normalem]{ulem}
\usepackage{amsmath}
\usepackage{textcomp}
\usepackage{amssymb}
\usepackage{capt-of}
\usepackage{hyperref}
\author{Vladmir Inđić}
\date{}
\title{Docker i Docker Compose}
\hypersetup{
 pdfauthor={Vladmir Inđić},
 pdftitle={Docker i Docker Compose},
 pdfkeywords={},
 pdfsubject={},
 pdfcreator={Emacs 27.1 (Org mode 9.4)}, 
 pdflang={English}}
\begin{document}

\maketitle
Današnji termin vežbi služi kao pregled alata za kontejnerizaciju i orkestraciju.
Studenti bi trebalo da se upoznaju sa radom sledećih alata:
\begin{itemize}
\item \href{https://github.com/vladaindjic/SCM-exchange-students\#docker}{Docker}
\item \href{https://github.com/vladaindjic/SCM-exchange-students\#docker-hub}{Docker Hub}
\item \href{https://github.com/vladaindjic/SCM-exchange-students\#docker-compose}{Docker Compose}.
\end{itemize}

Primeri upotrebe ovih alata mogu se pronaći na sledećim repozitorijumima:
\begin{itemize}
\item \href{https://github.com/vladaindjic/DjangoAuthTests}{Upotreba Docker i Docker Hub alata}
\item \href{https://github.com/vladaindjic/DjangoMimicProduction}{Upotreba Docker Compose alata}.
\end{itemize}
\end{document}
